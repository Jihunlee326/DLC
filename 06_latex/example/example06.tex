\documentclass[a4paper,11pt]{article}

\usepackage{kotex}

% define the title
\author{Il Gu Yi}
\title{Structure of text}
\date{\today}

\begin{document}
% generates the title
\maketitle

% insert the table of contents
\tableofcontents

\newpage
\section{나는 section}
\label{sec:this}
그냥 아무말이나 써야겠다.
이번 section~\ref{sec:this}은 그냥 아무말이나 써야겠다.
그냥 아무말이나 써야겠다.

\flushleft
\begin{enumerate}
  \item You can mix the list environments to your taste:
  \begin{itemize}
    \item But it might start to look silly.
    \item[-] With a dash.
  \end{itemize}
  \item Therefore remember:
  \begin{description}
    \item[Stupid] things will not
        become smart because they are in a list.
    \item[Smart] things, though, can be
        presented beautifully in a list.
  \end{description}
\end{enumerate}



\newpage
\section{나도 section}
\subsection{나도 subsection}
\subsubsection{나도 subsubsection}
그냥 아무말이나 써야겠다.
그냥 아무말이나 써야겠다.
이전 section~\ref{sec:this}은 page~\pageref{sec:this} 에 있다.
그냥 아무말이나 써야겠다.

\subsubsection*{나도 subsubsection}
그냥 아무말이나 써야겠다.
그냥 아무말이나 써야겠다.
그냥 아무말이나 써야겠다.


\paragraph{나는 paragraph}
blar blar blar blar blar blar blar blar
blar blar blar blar blar blar blar blar
blar blar blar blar blar blar blar blar

\subparagraph{나는 subparagraph}
blar blar blar blar blar blar blar blar
blar blar blar blar blar blar blar blar
blar blar blar blar blar blar blar blar

















\end{document}
