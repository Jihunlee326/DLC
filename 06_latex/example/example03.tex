\documentclass{article}

\begin{document}

"In" its most general form, convolution is an operation on two functions
of a real-valued argument.
''To motivate 'the' definition'' of convolution, we start with examples of
two functions we might use.

Suppose we are tracking the location of a spaceship with a laser sensor.
Our laser sensor provides a single output $x(t)$, the position of the
spaceship at time $t$.
Both $x$ and $t$ are real-valued, i.e., we can get a different reading
from the laser sensor at any instant in time.\\
Now suppose that our laser sensor is somewhat noisy.
To obtain a less noisy estimate of the spaceship’s position,
we would like to average together several measurements.
Of course, more recent measurements are more relevant, so we will
want this to be a weighted average that gives more weight to recent
measurements.
\newpage
We can do this with a weighting function $w(a)$, where a is the age of
a measurement.
If we apply such a weighted average operation at every moment, we obtain
a new function $s$ providing a smoothed estimate of the position
of the spaceship:

\end{document}
